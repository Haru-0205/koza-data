\documentclass{ltjsarticle}
\usepackage{graphicx}
\begin{document}

\title{inkscapeに関して}
\author{谷口 陽音}
\date{2023-07-09}
\maketitle

\section{オープンソース}
\subsection{オープンソースとは}
オープンソースとは、ソースコードが公開されていることをいう。\\
GitHubなどがよく使われる。
\subsection{オープンソース・プロジェクトの例}
\subsubsection{本プロジェクト}
本プロジェクトは、GitHubにより、すべてのソースコードが公開されている。
したがって、これは一種のオープンソース・プロジェクトと表現できる。
\subsubsection{Inkscape}
Inkscapeは、代表的なAdobe Illustratorの代替ソフトであると同時に、オープンソース・ベクター・グラフィックスの代表例である。\\
詳しくは後述する。
\subsubsection{GIMP}
GIMPは、"GNU Image Manipulation Program"の略で、オープンソースのAdobe Photoshopの代替ソフトとして有名である。
フリーでありながらPhotoshopに負けず劣らずな機能を備える。 
\subsubsection{GNU}
GNUは、"GNU Is not a Unix"の略(とされている)。UNIXというOSとのそ周辺アプリをオープンソース化するプロジェクトである。
GNUは後述するLinuxとも深いかかわりをもつ。
\subsubsection{Linux}
主催者が今使っているOS。狭義にはその核となる"Kernel"。\\
現在のWebの台頭は、Linuxのおかげともいえる。ちなみに、創始者のLinus Torvalds氏は、Gitの創始者でもある。
\subsection{フリーとは}
オープンソースを語るうえで欠かせない言葉である「フリー」。しかし、この言葉には複数の意味が込められている。
\subsubsection{英語の意味}
Free:
\begin{itemize}
    \item 無償の、無料の
    \item 自由な
\end{itemize}
\subsubsection{オープンソースの特徴}
\begin{itemize}
    \item 無償の
    \item 自由に(コードを)改変できる
    \item 自由に再配布できる
\end{itemize}
\subsection{「フリー」の実例}
\subsubsection{「無償」の実例}
\begin{itemize}
    \item オープンソース・ソフトウェアは無償で入手できる
\end{itemize}
\subsubsection{「自由に改変可能」の実例}
\begin{itemize}
    \item 「vi」というエディターのコードを改変して、さらに高機能なエディター「vim」が誕生した。
    \item Linux Kernelのコードを使って、GoogleがAndroidを開発した。
\end{itemize}
\subsubsection{「自由に再配布可能」の実例}
\begin{itemize}
    \item Linux Kernelだけだと使いにくいので、様々なソフトウェアと組み合わせて自分のページで公開する。
    \item いくつかのソフトウェアの組み合わせがよかったので、パッケージマネージャーから一括インストールできるようにリポジトリに追加した。
\end{itemize}
\subsection{オープンソース・ソフトウェアのライセンス}
オープンソースソフトウェアには様々な種類がある。以下に代表的なものを示す
\begin{itemize}
    \item GPL(GNU General Public License)
    \item MPL(Mozilla Public License)
\end{itemize}
\section{Inkscape}
\subsection{Inkscapeの特徴}
\subsubsection{オープンソース}
Inkscapeはオープンソースで開発されており、商用・非商用問わず誰でも自由に利用することができる。\\
これはInkscapeの最も大きな特徴の一つです。
\subsubsection{ベクター形式}
Inkscapeは、数少ないベクター・グラフィックス・スイートの1つである。\\
最も有名なベクター・グラフィックス・スイートはAdobe Illustratorであるが、サブスクリプションしかなく、かつそのライセンス料が極めて高い(通称「Adobe税」)。
よって、本プロジェクトではInkscapeを採用することとなった。\\
ベクター形式については後述するが、簡単に言うと「拡大しても劣化しない」画像である。\\
\subsubsection{クロスプラットフォーム}
オープンソースであるがゆえに様々なプラットフォームむけにビルドされている。\\
さまざまな形式の実行ファイルを網羅しているため、Windows,Mac問わず同様に使える。\\
また、このソフトウェアは主にLinux向けに作成されているため、ほかのWindowsソフトウェアと操作感が若干違う。
<<<<<<< HEAD
\subsection{ベクター画像とラスター画像}
\subsubsection{ラスター形式とは}
ラスター形式とは、画像を小さいドットの集まりとして捉える形式である。
主にイラストや写真に用いられる。
\subsubsection{ベクター形式とは}
ベクター形式とは、画像を点や線の集合と捉え、それぞれについてパスと長さなどを用いて表す形式である。主にロゴ制作や印刷物制作に用いられる。
\subsubsection{ラスター形式の特徴}
\begin{itemize}
    \item メリット
	\subitem 線の太さの変化や濃淡を表現しやすい
	\subitem 様々なソフトウェアで表示や編集ができる
    \item デメリット
	\subitem 拡大すると粗くなる
    \item 代表的なフォーマット
	\subitem .jpg/.jpeg
	\subitem .png
	\subitem .bmp
\end{itemize}
\subsubsection{ベクター形式の特徴}
\begin{itemize}
    \item メリット
	\subitem 拡大しても画質劣化しない
    \item デメリット
	\subitem 読み書きがしにくい
	\subitem 濃淡や太さの変化は苦手
    \item 代表的なフォーマット
	\subitem .svg
	\subitem .pdf
\end{itemize}
\end{document}
