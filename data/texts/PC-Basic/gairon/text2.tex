\documentclass{ltjsarticle}
\begin{document}
\title{PC基礎講習テキスト\\vol.1}
\author{谷口 陽音}
\date{2023-07-16}
\maketitle
\section{OSについて}
\subsection{OSとは}
OS[Operating System]は、一般に基本ソフトウェアのことを指し、これはPCが動作するために必要なものである。\\
OSにもさまざまな種類があり、さまざまなシステムに組み込まれる。\\
OSは主にC言語で書かれている。また、OSによって実行ファイルの形式が異なる。\\
実行ファイルについては後述する。
\subsection{OSの例}
ここでは、代表的なOSを列挙する。
\begin{itemize}
    \item PC
    \begin{itemize}
        \item Windows
        \item MacOS
        \item ChromeOS
        \item Linux\footnote[1]{詳しい種類(ディストリビューション)については後述}
        \item FreeBSD
    \end{itemize}
    \item スマートフォン・タブレット
    \begin{itemize}
        \item Android
        \item iOS
        \item iPadOS
        \item Windows Phone
    \end{itemize}
    \item サーバー
    \begin{itemize}
        \item Linux
        \item Windows Server
    \end{itemize}
\end{itemize}
\subsection{実行ファイル}
実行ファイルとは、そのOS上で実行可能なファイルのことを指す。\\
狭義には、アプリケーション・ソフトウェアの実体となるファイルを指す。\\
表1に代表的なOSとその実行ファイルの拡張子を示す。\\

\begin{table}
    \caption{主なOSと実行ファイル拡張子}
    \begin{center}
\begin{tabular}{|l|l|} \hline
    OS & 拡張子 \\ \hline \hline
    Windows & .exe,.msi \\ \hline
    MacOS & .app,.ipa \\ \hline
    Android & .apk \\ \hline
    iOS,iPadOS & .ipa \\ \hline
    Linux & .appimage,.sh\footnote[2]{Linuxは若干特殊なため、詳しくは後述} \\ \hline
\end{tabular}
\end{center}
\end{table}
\subsection{Windows}
Windowsは、Microsoft社が開発したOSで、クライアント向けの「Windows」シリーズとサーバー向けの「Windows Server」がある。\\
Windows Subsystem for Linuxを用いることでLinux環境を構築できる。\\
Windowsのシェア率は非常に高く、70\%を超えている。\\
なお、標準ブラウザーであるChromium系のMicrosoft Edgeは今ひとつと言われている。
\subsection{MacOS}
MacOSは、Apple社が開発したUNIX系のOSで、Apple純正コンピューター「Mac」シリーズ以外には搭載できない\footnote[3]{そのMacOSを他のデバイスに移植しようとするHackintoshという試みもあるが、これはAppleのライセンス違反となるため、推奨しない。}。\\
UNIX系のため、Webデザインなどによく使われ、デザイン業界のスタンダードである。\\
iPhoneやiPadといった他のApple製品と連携させることができる。\\
標準ブラウザーはChromium系のSafariであり、これもApple製OSでないと動かない。
\subsection{ChromeOS}
ChromeOSは、Googleが開発したWebベースのOSである。\\
インターネット接続が前提とされたOSであり、RAM容量およびROM容量は少なくて良い。\\
純正ChromeOSはGoogle Play StoreでAndroidアプリのインストールが可能である。\\
なお、これのオープンソース版をChromium OSという。
\subsection{iOS/iPadOS}
iOSおよびiPadOSは、Apple社が開発したモバイル向けOSである。\\
これらのOSは、アプリストア「AppStore」経由でしかソフトウェアがインストールできない。\\
また、これらのOS上で動作するアプリケーションについては、Macでしか開発できない。\\
\subsection{Android}
Androidは、GoogleがLinuxをカスタマイズしてモバイルプラットフォームに最適化したモバイル向けOSである。\\
iOS/iPadOSとは違い、Google Play Store以外のソースからのアプリもインストール可能である。\\
また、AndroidアプリはMacだけでなく、WindowsやLinuxでも開発できる。\\
なお、Androidアプリ開発用ソフト「Android Studio」はJetBrainsのIntelliJ IDEA Communityがベースになっており、開発言語もまたJetBrains製のKotlinが推奨されている。
\subsection{Linux}
\subsubsection{Linuxとは}
Linuxとは、伝説的エンジニアLinus TorvaldsがUnixやMinix\footnote[4]{当時よく使われていた教育向けのオープンソースなOS}をもとに開発したOSS\footnote[4]{Open Source Software:オープンソースソフトウェア(ソースコードを公開して開発すること)}である。\\
今日のWebの普及はLinuxがあってこそのものだともいわれている。\\
また、先述したように、Googleがソースコードを改変してAndroidを作ったように、OSSは改変や再配布が自由に可能である。\\
そもそもGUIを持っていないことも多いが、GUIを持っているものについては、デフォルトブラウザーはFireFoxやChromiumといったOSSが採用されている。
\subsubsection{KernelとDistribution}
狭義のLinuxは「Kernel」と呼ばれる「核」にあたる部分のみを示す。GoogleがAndroidを作ったときにベースにしたのもこのKernelである。\\
しかし、KernelだけではOSは動かず、他にShellなどのソフトウェアが必要で、LinusはGNUプロジェクトのソフトウェアを自力でインストールすることを前提としていた。\\
そこで、Kernelとその他動作に必要なソフト類をまとめて配布するパッケージが誕生した。これは、「Linuxディストリビューション」とよばれ、広義のLinuxはこれを含むうえ、一般にLinuxというとLinuxディストリビューションのことを指す。また、Linuxディストリビューションは、「ディストロ」と略されることも多い。
いまでは、Linuxディストリビューションは星の数ほど存在し、日々新しいプロジェクトが立ち上がっては、プロジェクトが消滅していっている。
\end{document}