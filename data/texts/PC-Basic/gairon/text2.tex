\documentclass{ltjsarticle}
\begin{document}
\title{PC基礎講習テキスト\\vol.1}
\author{谷口 陽音}
\date{2023-07-16}
\maketitle
\section{OSについて}
\subsection{OSとは}
OS[Operating System]は、一般に基本ソフトウェアのことを指し、これはPCが動作するために必要なものである。\\
OSにもさまざまな種類があり、さまざまなシステムに組み込まれる。\\
OSは主にC言語で書かれている。また、OSによって実行ファイルの形式が異なる。\\
実行ファイルについては後述する。
\subsection{OSの例}
ここでは、代表的なOSを列挙する。
\begin{itemize}
    \item PC
    \begin{itemize}
        \item Windows
        \item MacOS
        \item ChromeOS
        \item Linux\footnote[1]{詳しい種類(ディストリビューション)については後述}
        \item FreeBSD
    \end{itemize}
    \item スマートフォン・タブレット
    \begin{itemize}
        \item Android
        \item iOS
        \item iPadOS
        \item Windows Phone
    \end{itemize}
    \item サーバー
    \begin{itemize}
        \item Linux
        \item Windows Server
    \end{itemize}
\end{itemize}
\subsection{実行ファイル}
実行ファイルとは、そのOS上で実行可能なファイルのことを指す。\\
狭義には、アプリケーション・ソフトウェアの実体となるファイルを指す。\\
表1に代表的なOSとその実行ファイルの拡張子を示す。\\

\begin{table}
    \caption{主なOSと実行ファイル拡張子}
    \begin{center}
\begin{tabular}{|l|l|} \hline
    OS & 拡張子 \\ \hline \hline
    Windows & .exe,.msi \\ \hline
    MacOS & .app,.ipa \\ \hline
    Android & .apk \\ \hline
    iOS,iPadOS & .ipa \\ \hline
    Linux & .appimage,.sh\footnote[2]{Linuxは若干特殊なため、詳しくは後述} \\ \hline
\end{tabular}
\end{center}
\end{table}
\subsection{Windows}
Windowsは、Microsoft社が開発したOSで、クライアント向けの「Windows」シリーズとサーバー向けの「Windows Server」がある。\\
Windows Subsystem for Linuxを用いることでLinux環境を構築できる。\\
Windowsのシェア率は非常に高く、70\%を超えている。\\
なお、標準ブラウザーであるChromium系のMicrosoft Edgeは今ひとつと言われている。
\subsection{MacOS}
MacOSは、Apple社が開発したUNIX系のOSで、Apple純正コンピューター「Mac」シリーズ以外には搭載できない\footnote[3]{そのMacOSを他のデバイスに移植しようとするHackintoshという試みもあるが、これはAppleのライセンス違反となるため、推奨しない。}。\\
UNIX系のため、Webデザインなどによく使われ、デザイン業界のスタンダードである。\\
iPhoneやiPadといった他のApple製品と連携させることができる。\\
標準ブラウザーはChromium系のSafariであり、これもApple製OSでないと動かない。
\subsection{ChromeOS}
ChromeOSは、Googleが開発したWebベースのOSである。\\
インターネット接続が前提とされたOSであり、RAM容量およびROM容量は少なくて良い。\\
純正ChromeOSはGoogle Play StoreでAndroidアプリのインストールが可能である。\\
なお、これのオープンソース版をChromium OSという。
\subsection{iOS/iPadOS}
iOSおよびiPadOSは、Apple社が開発したモバイル向けOSである。\\
これらのOSは、アプリストア「AppStore」経由でしかソフトウェアがインストールできない。\\
また、これらのOS上で動作するアプリケーションについては、Macでしか開発できない。\\
\subsection{Android}
Androidは、GoogleがLinuxをカスタマイズしてモバイルプラットフォームに最適化したモバイル向けOSである。\\
iOS/iPadOSとは違い、Google Play Store以外のソースからのアプリもインストール可能である。\\
また、AndroidアプリはMacだけでなく、WindowsやLinuxでも開発できる。\\
なお、Androidアプリ開発用ソフト「Android Studio」はJetBrainsのIntelliJ IDEA Communityがベースになっており、開発言語もまたJetBrains製のKotlinが推奨されている。
\subsection{Linux}
\subsubsection{Linuxとは}
Linuxとは、伝説的エンジニアLinus TorvaldsがUnixやMinix\footnote[4]{当時よく使われていた教育向けのオープンソースなOS}をもとに開発したOSS\footnote[4]{Open Source Software:オープンソースソフトウェア(ソースコードを公開して開発すること)}である。\\
今日のWebの普及はLinuxがあってこそのものだともいわれている。\\
また、先述したように、Googleがソースコードを改変してAndroidを作ったように、OSSは改変や再配布が自由に可能である。\\
そもそもGUIを持っていないことも多いが、GUIを持っているものについては、デフォルトブラウザーはFireFoxやChromiumといったOSSが採用されている。
\subsubsection{KernelとDistribution}
狭義のLinuxは「Kernel」と呼ばれる「核」にあたる部分のみを示す。GoogleがAndroidを作ったときにベースにしたのもこのKernelである。\\
しかし、KernelだけではOSは動かず、他にShellなどのソフトウェアが必要で、LinusはGNUプロジェクトのソフトウェアを自力でインストールすることを前提としていた。\\
そこで、Kernelとその他動作に必要なソフト類をまとめて配布するパッケージが誕生した。これは、「Linuxディストリビューション(Distribution)」とよばれ、広義のLinuxはこれを含むうえ、一般にLinuxというとLinuxディストリビューションのことを指す。また、Linuxディストリビューションは、「ディストロ」と略されることも多い。
いまでは、Linuxディストリビューションは星の数ほど存在し、日々新しいプロジェクトが立ち上がっては、プロジェクトが消滅していっている。
\subsubsection{Distributionの分類}
現在ディストリビューションは星の数ほど存在し、分類をしなければ特徴を掴むのは極めて難しい。\\
よって、表2のような分類がある。なお、パッケージマネージャーについては後述する。
\begin{table}
    \caption{ディストリビューションの分類}
    \begin{center}
        \begin{tabular}{lll}\hline
            系統 & 特徴 & パッケージマネージャー \\ \hline \hline
            Debian系 & 最も多くのディストリビューションを擁する & apt \\ \hline
            RedHat系 & 元となったディストロは商用シェアNo.1 & rpm \\ \hline
            Arch系 & 安定性が高く、コミュニティが活発 & pacman \\ \hline
            独立系 & 上記の3つのいずれにも属さないディストロ & ディストロによる \\ \hline
        \end{tabular}
    \end{center}
\end{table}
\subsubsection{ユーザーインターフェース}
ユーザーインターフェースは、略してUIともよばれる、私達がコンピュータを操作するときにみる画面のことである。\\
Linuxにおいては、GUIとCLIの2つに大別される。\\
GUIは、Graphical User Interfaceの略で、WindowsやMacOSのように、キーボードとマウスなどを用いて操作できる形式である。\\
一方、CLIはCommand Line Interfaceの略で、コマンド環境のみで操作する形式のことである。\\
CUI:Character User Interface,Character-based User Interface,またはCommand User InterfaceもCLIと同義である。
\subsubsection{シェル}
CLI環境に置いては、コマンドの入力インターフェース兼コマンド入力補助ツールとして「シェル」とよばれるプログラムが使用される。\\
Windowsでは「コマンドプロンプト」や「Windows PowerShell」などに当たる。\\
おそらく一番有名なシェルは「Bash」である。
\subsubsection{デスクトップ環境}
GUI環境をもつLinuxは、事実上デスクトップ環境(略称:DE)を持つ。\\
表3に主なデスクトップ環境を挙げる。なお、谷口が今気に入っているのはKDE Plasmaである。\\
\begin{table}
    \caption{主なデスクトップ環境}
    \begin{center}
        \begin{tabular}{lll} \hline
            DE名 & 特徴 & 開発元 \\ \hline \hline
            GNOME Shell & 3D効果を駆使したグラフィカルなDE & GNU \\ \hline
            KDE Plasma & 美しく、モダンで高機能性なDE & KDE \\ \hline
            Xfce & 軽さと機能性を両立したスマートなDE \\ \hline
            LXDE/LXQt & 軽さを追求したとにかく軽いDE \\ \hline
            Budgie & すっきりとしたデザインでモダンなDE \\ \hline
        \end{tabular}
    \end{center}
\end{table}
\subsubsection{Linuxで利用するソフト}
もちろんLinuxでも十分実用に耐えうるようなソフトウェアが用意されている。\\
なお、この中にはOSSであったり、OSSでなくてもWindowsやMacでも利用可能なソフトが含まれる。このようなソフトウェアを「クロス・プラットフォーム」という。\\
あまり知名度が高くないソフトウェアも多いが、本プロジェクトでは基本的にこの中にあるようなソフトウェアを活用する。\\
今回活用する予定のクロス・プラットフォームのソフトを表4に掲載する。
\begin{table}
    \caption{本プロジェクトで利用予定のソフトウェア}
    \begin{center}
        \begin{tabular}{lll}\hline
            ソフト名 & 用途 & 類似ソフト \\ \hline \hline
            Inkscape & ベクター・グラフィックス & Adobe Illustrator \\ \hline
            GIMP & 画像編集 & Adobe Photoshop \\ \hline
            Krita & ラスター・グラフィックス & Clip Studio Paint \\ \hline
            Microsoft Visual Studio Code & テキストエディタ \\ \hline
            GitHub Desktop & Git管理 \\ \hline
            Figma & UI設計 \\ \hline
        \end{tabular}
    \end{center}
\end{table}
\subsubsection{コマンドライン・ツール}
コマンド上から実行するツールのことをコマンドライン・ツールと言う。\\
私はLinuxを常用している関係上、コマンドライン・ツールをよく利用する。\\
以下にその一例を挙げる
\begin{itemize}
    \item Vim/NeoVim
    \item git
    \item GitHub CLI
    \item LilyPond
    \item Windows File Recovery
    \item LaTeX
\end{itemize}
\subsection{まとめ}
\begin{itemize}
    \item OSはコンピュータの動作に必要不可欠
    \item 一口にOSといっても様々なものがある
    \item 実際の操作画面のことをUIという
    \item UIはGUIとCUIという2種類に大別される
    \item 本プロジェクトでは基本的にOSSでクロスプラットフォームなアプリケーションを利用する
\end{itemize}
\end{document}