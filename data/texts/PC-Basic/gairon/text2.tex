\documentclass{ltjsarticle}
\begin{document}
\title{PC基礎講習テキスト\\vol.1}
\author{谷口 陽音}
\date{2023-07-16}
\maketitle
\section{OSについて}
\subsection{OSとは}
OS[Operating System]は、一般に基本ソフトウェアのことを指し、これはPCが動作するために必要なものである。\\
OSにもさまざまな種類があり、さまざまなシステムに組み込まれる。\\
OSは主にC言語で書かれている。また、OSによって実行ファイルの形式が異なる。\\
実行ファイルについては後述する。
\subsection{OSの例}
ここでは、代表的なOSを列挙する。
\begin{itemize}
    \item PC
    \begin{itemize}
        \item Windows
        \item MacOS
        \item ChromeOS
        \item Linux\footnote[1]{詳しい種類(ディストリビューション)については後述}
        \item FreeBSD
    \end{itemize}
    \item スマートフォン・タブレット
    \begin{itemize}
        \item Android
        \item iOS
        \item iPadOS
        \item Windows Phone
    \end{itemize}
    \item サーバー
    \begin{itemize}
        \item Linux
        \item Windows Server
    \end{itemize}
\end{itemize}
\subsection{実行ファイル}
実行ファイルとは、そのOS上で実行可能なファイルのことを指す。\\
狭義には、アプリケーション・ソフトウェアの実体となるファイルを指す。\\
表1に代表的なOSとその実行ファイルの拡張子を示す。\\

\begin{table}
    \caption{主なOSと実行ファイル拡張子}
    \begin{center}
\begin{tabular}{|l|l|} \hline
    OS & 拡張子 \\ \hline \hline
    Windows & .exe,.msi \\ \hline
    MacOS & .app,.ipa \\ \hline
    Android & .apk \\ \hline
    iOS,iPadOS & .ipa \\ \hline
    Linux & .appimage,.sh\footnote[2]{Linuxは若干特殊なため、詳しくは後述} \\ \hline
\end{tabular}
\end{center}
\end{table}
\subsection{Windows}
Windowsは、Microsoft社が開発したOSで、クライアント向けの「Windows」シリーズとサーバー向けの「Windows Server」がある。\\
Windows Subsystem for Linuxを用いることでLinux環境を構築できる。\\
Windowsのシェア率は非常に高く、70\%を超えている。\\
なお、標準ブラウザーであるChromium系のMicrosoft Edgeは今ひとつと言われている。
\subsection{MacOS}
MacOSは、Apple社が開発したUNIX系のOSで、Apple純正コンピューター「Mac」シリーズ以外には搭載できない\footnote[3]{そのMacOSを他のデバイスに移植しようとするHackintoshという試みもあるが、これはAppleのライセンス違反となるため、推奨しない。}。\\
UNIX系のため、Webデザインなどによく使われ、デザイン業界のスタンダードである。\\
iPhoneやiPadといった他のApple製品と連携させることができる。\\
標準ブラウザーはChromium系のSafariであり、これもApple製OSでないと動かない。
\subsection{ChromeOS}
ChromeOSは、Googleが開発したWebベースのOSである。\\
インターネット接続が前提とされたOSであり、RAM容量およびROM容量は少なくて良い。\\
純正ChromeOSはGoogle Play StoreでAndroidアプリのインストールが可能である。\\
なお、これのオープンソース版をChromium OSという。
\subsection{iOS/iPadOS}
iOSおよびiPadOSは、Apple社が開発したモバイル向けOSである。\\
これらのOSは、アプリストア「AppStore」経由でしかソフトウェアがインストールできない。\\
また、これらのOS上で動作するアプリケーションについては、Macでしか開発できない。\\
\subsection{Android}
Androidは、GoogleがLinuxをカスタマイズしてモバイルプラットフォームに最適化したモバイル向けOSである。\\
iOS/iPadOSとは違い、Google Play Store以外のソースからのアプリもインストール可能である。\\
また、AndroidアプリはMacだけでなく、WindowsやLinuxでも開発できる。\\
なお、Androidアプリ開発用ソフト「Android Studio」はJetBrainsのIntelliJ IDEA Communityがベースになっており、開発言語もまたJetBrains製のKotlinが推奨されている。
\subsection{Linux}
\subsubsection{Linuxとは}
Linuxとは、伝説的エンジニアLinus TorvaldsがUnixやMinix\footnote[4]{当時よく使われていた教育向けのオープンソースなOS}をもとに開発したOSS\footnote[4]{Open Source Software:オープンソースソフトウェア(ソースコードを公開して開発すること)}である。\\
今日のWebの普及はLinuxがあってこそのものだともいわれている。\\
また、先述したように、Googleがソースコードを改変してAndroidを作ったように、OSSは改変や再配布が自由に可能である。\\
そもそもGUIを持っていないことも多いが、GUIを持っているものについては、デフォルトブラウザーはFireFoxやChromiumといったOSSが採用されている。
\subsubsection{KernelとDistribution}
狭義のLinuxは「Kernel」と呼ばれる「核」にあたる部分のみを示す。GoogleがAndroidを作ったときにベースにしたのもこのKernelである。\\
しかし、KernelだけではOSは動かず、他にShellなどのソフトウェアが必要で、LinusはGNUプロジェクトのソフトウェアを自力でインストールすることを前提としていた。\\
そこで、Kernelとその他動作に必要なソフト類をまとめて配布するパッケージが誕生した。これは、「Linuxディストリビューション(Distribution)」とよばれ、広義のLinuxはこれを含むうえ、一般にLinuxというとLinuxディストリビューションのことを指す。また、Linuxディストリビューションは、「ディストロ」と略されることも多い。
いまでは、Linuxディストリビューションは星の数ほど存在し、日々新しいプロジェクトが立ち上がっては、プロジェクトが消滅していっている。
\subsubsection{Distributionの分類}
現在ディストリビューションは星の数ほど存在し、分類をしなければ特徴を掴むのは極めて難しい。\\
よって、表2のような分類がある。なお、パッケージマネージャーについては後述する。
\begin{table}
    \caption{ディストリビューションの分類}
    \begin{center}
        \begin{tabular}{lll}\hline
            系統 & 特徴 & パッケージマネージャー \\ \hline \hline
            Debian系 & 最も多くのディストリビューションを擁する & apt \\ \hline
            RedHat系 & 元となったディストロは商用シェアNo.1 & rpm \\ \hline
            Arch系 & 安定性が高く、コミュニティが活発 & pacman \\ \hline
            独立系 & 上記の3つのいずれにも属さないディストロ & ディストロによる \\ \hline
        \end{tabular}
    \end{center}
\end{table}
\subsubsection{ユーザーインターフェース}
ユーザーインターフェースは、略してUIともよばれる、私達がコンピュータを操作するときにみる画面のことである。\\
Linuxにおいては、GUIとCLIの2つに大別される。\\
GUIは、Graphical User Interfaceの略で、WindowsやMacOSのように、キーボードとマウスなどを用いて操作できる形式である。\\
一方、CLIはCommand Line Interfaceの略で、コマンド環境のみで操作する形式のことである。\\
CUI:Character User Interface,Character-based User Interface,またはCommand User InterfaceもCLIと同義である。
\subsubsection{シェル}
CLI環境に置いては、コマンドの入力インターフェース兼コマンド入力補助ツールとして「シェル」とよばれるプログラムが使用される。\\
Windowsでは「コマンドプロンプト」や「Windows PowerShell」などに当たる。\\
おそらく一番有名なシェルは「Bash」である。
\subsubsection{デスクトップ環境}
GUI環境をもつLinuxは、事実上デスクトップ環境(略称:DE)を持つ。\\
表3に主なデスクトップ環境を挙げる。なお、谷口が今気に入っているのはKDE Plasmaである。\\
\begin{table}
    \caption{主なデスクトップ環境}
    \begin{center}
        \begin{tabular}{lll} \hline
            DE名 & 特徴 & 開発元 \\ \hline \hline
            GNOME Shell & 3D効果を駆使したグラフィカルなDE & GNU \\ \hline
            KDE Plasma & 美しく、モダンで高機能性なDE & KDE \\ \hline
            Xfce & 軽さと機能性を両立したスマートなDE \\ \hline
            LXDE/LXQt & 軽さを追求したとにかく軽いDE \\ \hline
            Budgie & すっきりとしたデザインでモダンなDE \\ \hline
        \end{tabular}
    \end{center}
\end{table}
\subsubsection{Linuxで利用するソフト}
もちろんLinuxでも十分実用に耐えうるようなソフトウェアが用意されている。\\
なお、この中にはOSSであったり、OSSでなくてもWindowsやMacでも利用可能なソフトが含まれる。このようなソフトウェアを「クロス・プラットフォーム」という。\\
あまり知名度が高くないソフトウェアも多いが、本プロジェクトでは基本的にこの中にあるようなソフトウェアを活用する。\\
今回活用する予定のクロス・プラットフォームのソフトを表4に掲載する。
\begin{table}
    \caption{本プロジェクトで利用予定のソフトウェア}
    \begin{center}
        \begin{tabular}{lll}\hline
            ソフト名 & 用途 & 類似ソフト \\ \hline \hline
            Inkscape & ベクター・グラフィックス & Adobe Illustrator \\ \hline
            GIMP & 画像編集 & Adobe Photoshop \\ \hline
            Krita & ラスター・グラフィックス & Clip Studio Paint \\ \hline
            Microsoft Visual Studio Code & テキストエディタ \\ \hline
            GitHub Desktop & Git管理 \\ \hline
            Figma & UI設計 \\ \hline
        \end{tabular}
    \end{center}
\end{table}
\subsubsection{コマンドライン・ツール}
コマンド上から実行するツールのことをコマンドライン・ツールと言う。\\
私はLinuxを常用している関係上、コマンドライン・ツールをよく利用する。\\
以下にその一例を挙げる
\begin{itemize}
    \item Vim/NeoVim
    \item git
    \item GitHub CLI
    \item LilyPond
    \item Windows File Recovery
    \item LaTeX
\end{itemize}
\subsection{まとめ}
\begin{itemize}
    \item OSはコンピュータの動作に必要不可欠
    \item 一口にOSといっても様々なものがある
    \item 実際の操作画面のことをUIといい、GUIとCUIという2種類に大別される
    \item 本プロジェクトでは基本的にOSSでクロスプラットフォームなアプリケーションを利用する
\end{itemize}
\section{ソフトウェアについて}
\subsection{ソフトウェアとは}
ソフトウェアは、基本ソフトウェア(=OS)と応用ソフトウェア(=アプリケーション)に大別される。\\
基本ソフトウェアについては先述の通りであるため、ここでは応用ソフトウェアおよびソフトウェアの動作原理・プログラミングについて触れる。
\subsection{ソフトウェアの構成と作成方法}
ソフトウェアは、プログラムとライブラリ・バッチファイルなどをまとめてパッケージ化して配布したものと捉えることができる。\\
プログラムは無論プログラミング言語を用いて作成される。大抵の場合次のような手順を踏む。
\begin{enumerate}
    \item 計画
    \item プログラミング
    \item コンパイル
    \item ビルド
    \item 配布
\end{enumerate}
以下では特にプログラミングに焦点を置く。
\subsection{プログラミングとは}
プログラミングとは、コンピュータへの命令を書いたファイルを作成する一連の作業のことである。\\
ここで作成したファイルのことをプログラムと呼ぶ。\\
混同されがちなものにシェルスクリプトがあるが、シェルスクリプトはプログラムの1つと捉えることができる。
\subsection{プログラミング言語}
プログラミングに用いる言語をプログラミング言語というが、これらは以下の2つに大別させる。
\begin{itemize}
    \item コンパイラ言語
    \item スクリプト言語
\end{itemize}
それぞれに利点があるが、ここでは主にスクリプト言語を扱う
\subsubsection{このプロジェクトで扱うプログラミング言語}
本プロジェクトは、主にWebを対象とするため、自ずと利用する言語は限られてくる。
以下にそれぞれの特徴を示す
\subsubsection{JavaScript}
スクリプト言語で、主にWeb開発に使われる。\\
Webブラウザ上で実行できるため、殆どの場合実行環境構築はしなくて良い。\\
なお、Node.jsとよばれるパッケージを用いることで、ローカル環境でも実行でき、Electronというツールを使えば、デスクトップアプリも作成可能と、その活躍幅はWebだけに留まらない。
\subsubsection{Python}
スクリプト言語で、様々な用途に使われる。\\
Web開発の中でもサーバーサイドの開発であるバックエンド開発によく用いられる。\\
AIなどにも使われるため、身につけて損はないと思うが、インデントが意味を持つ珍しいタイプの言語である。
\subsubsection{PHP}
スクリプト言語で、Web開発に使われる。\\
今回はJavaScriptに統一する予定だが、もしかしたら使うかもしれない。\\
JavaScriptとは違い、サーバーサイドで実行される。
\subsubsection{HTML/CSS}
正確にはプログラミング言語ではない。\\
HTMLはマークアップ言語、CSSはスタイルシートである。\\
Webページの枠組みや見た目を作るための言語である。
\subsection{実際にアプリケーション作成によく使われている言語}
OSによって大きく異なるため、OSごとに記す
\subsubsection{Windows}
Windowsのアプリ開発はわりと自由度が高いが、Microsoft社が開発したC\#を用いることが多い。\\
C\#はC言語系のJavaおよびC++の派生に位置付けられ、比較的シンプルに記述できる。\\
C\#以外では、C++やVisual Basic,Java,JavaScriptなどが用いられる。\\
なお、Windows開発で用いる場合、C\#,C++などはMicrosoftの「方言」があるため、Visualを付してよばれる。\\
また、C\#,VB.NET,F\#をまとめて.NETとよばれる。\\
IDEはMicrosoft Visual Studioを使うことが多い。
\subsubsection{MacOS,iOS,iPadOS}
Apple製品のアプリ開発は事実上SwiftまたはObjective-C系統の2択である。(正確にはMacではJavaScriptやJavaも動く)
また、Swiftの標準IDEはMac専用の「XCode」である。\\
UnityなどでWindowsでビルドはできるが、Macでリビルドしなければ動かせない。
\subsubsection{Android}
Androidは、少々前まではJavaを使っていたが、Javaの有償化に伴い、Kotlinが推奨されるようになった。\\
KotlinはチェコのJetBrains社が開発した言語である。\\
Androidの標準IDEであるAndroidStudioはJetBrains社のIntelliJ IDEA Communityがベースになっている。\\
なお、Android StudioはWindows,Mac,Linux対応である。
\section{Markdown}
本プロジェクトでは、標準ドキュメント形式としてMarkdownおよび\LaTeX を採用する。\\
そのうち、みなさんにはMarkdownを活用できるようになっていただきたい。\\
\subsection{Markdownとは}
Markdownは、軽量マークアップ言語の1つで、GitHubやQiita、Zennなどのエンジニア向けツールによく採用されている。\\
ただのプレーンテキストを少しの記号で見やすくすることができる。\\
これはメモとしても優秀で、構造化して記録できるため、非常に可読性が高いメモになる。
\subsection{MarkdownとHTML}
MarkdownはHTMLと互換性があり、Markdown内ではHTMLとMarkdownの共存が可能である。\\
そもそもMarkdownはHTMLを簡略化したものであり、Markdownで実装されなかったHTMLの機能をHTMLのタグを用いて表現できる。
\subsection{Mermaid}
Markdownでは、Mermaid記法というものを用いて、図をかける。\\
一部のエディターは非対応であるが、多くのエディターで表示できる。
\subsection{PlantUML}
Markdownでは、PlantUMLを用いてシーケンス図などを作成できる。\\
しかし、Mermaidよりも対応しているものは控えめである。
\subsection{記法}
Markdownでは、記号を用いて構造化する。
\subsubsection{見出し}
Markdownでは、HTML同様、第1レベル~第6レベルの見出しを設定できる。
\begin{itemize}
    \item \# 第1レベル
    \item \#\# 第2レベル
    \item \#\#\# 第3レベル
    \item \#\#\#\# 第4レベル
    \item \#\#\#\#\# 第5レベル
    \item \#\#\#\#\#\# 第6レベル
\end{itemize}
\subsubsection{箇条書き}
Markdownでは、よく使う記法である箇条書きが簡単にできる。
\begin{itemize}
    \item - ハイフン
    \item * アスタリスク
    \item \_ アンダーバー(アンダースコア)
\end{itemize}
のいずれかで箇条書きにできる
\subsubsection{番号付きリスト}
番号付きリストも作成できる
\begin{itemize}
    \item 1. 1つ目
    \item 2. 2つ目
    \item 1. 3つ目 番号は関係なく通し番号が振り直される
\end{itemize}
\subsubsection{表}
表は比較的簡単に作成可能
| 列1 | 列2 |
| --- | :---: |
| 内容 | 中央揃えの列 |
\subsection{拡張子}
Markdownファイルの拡張子は[.md]または[.markdown]である。
\section{コマンド操作とGit}
ここでは、Windows PowerShellの使い方とGitについて扱う
\subsection{コマンドと引数}
コマンド環境下に打ち込んで端末を制御する文字列をコマンドという。\\
また、コマンドのあとに文字列をつけることもあり、その文字のことを「引数」という。\\
読み方は「ひきすう」であり、これは数学の「因数」と区別するためにそうよばれる。
\subsection{基本コマンド}
PowerShell上でのファイル操作コマンドについて説明する
\subsubsection{ls}
ls[list]コマンドは、今いるフォルダー[カレントディレクトリ]内のファイルの一覧を表示するコマンドである。\\
引数にディレクトリパスを指定すれば、そのディレクトリの中身を表示する。
\subsubsection{pwd}
pwd[Print Working Directly]コマンドは、今いるディレクトリのパスを表示するコマンドである。\\
実行すると、現在のカレントディレクトリの絶対パスが出力される。
\subsubsection{mkdir}
mkdir[MaKe DIRectly]コマンドは、カレントディレクトリ以下にディレクトリを作成するコマンドである。\\
ディレクトリ名を引数に取る。
\subsubsection{cd}
cd[Change Directly]コマンドは、カレントディレクトリを移動するコマンドである。\\
移動先を引数に取り、パスは相対パスでも絶対パスでも可。
\subsection{gitのセットアップ}
Gitをインストールしたあとには少々設定が必要である。\\
そのときに使うのはGitコマンドだ。\\
\subsubsection{git config}
git configコマンドは、主にGitを使うための設定を行うコマンドである。\\
Gitを使うために必要な設定コマンドをいかに示す
\begin{itemize}
    \item git config --global user.email <email>
    \item git config --global user.name <name>
\end{itemize}
以上のコマンドは、Gitを使用するにあたって、コミットする際に必要なユーザー名とメールアドレスを登録するコマンドである。
\subsubsection{git clone}
git cloneコマンドは、リモートリポジトリをローカルにクローンするコマンドである。\\
Gitで一番基本的なコマンドであり、始めるために必要不可欠なコマンドと言ってもいいだろう\\
構文を以下に挙げる。
\begin{itemize}
    \item git clone <url>
\end{itemize}
ここでURLは、GitHubの場合、[https://github.com/<name>/<repo>]である。\\
たとえば、本プロジェクトのソースをクローンしたい場合、「git clone https://github.com/haru-0205/OMUCT\_1-4」と入力する。
\section{次回について}
次回は、Markdown、今回のコマンドの続きとハードウェアについて行う。\\
特にコマンドでは、パッケージのインストールを行うコマンドを実際に実行してもらう。\\
それに伴い、(殆どの場合インストールされてると思うが、)以下のソフトウェアを事前にインストールしておくこと。
\begin{itemize}
    \item Windows Package Manager (WinGet)
\end{itemize}
Microsoft Storeから簡単に導入できる。\\


なお、次回導入するソフトウェアは下記の通りである。\\
可能な限り導入は避けること。
\begin{itemize}
    \item Obsidian
    \item GitHub CLI
    \item ImageMagic
    \item GitKraken
\end{itemize}

また、次回の講習では、GitHubアカウントが必須なので、準備しておくこと。
\end{document}